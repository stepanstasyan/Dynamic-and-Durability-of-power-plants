\documentclass[12pt,a4paper]{article}
\usepackage[utf8]{inputenc}
\usepackage{amsmath}
\usepackage{amsfonts}
\usepackage{amssymb}
\usepackage{textcomp}
\usepackage{siunitx}
\usepackage{multirow}

\usepackage{setspace}




    \usepackage{cmap} % для кодировки шрифтов в pdf
    \usepackage[T2A]{fontenc}
    \usepackage[russian]{babel}
    \usepackage{graphicx} % для вставки картинок
    \usepackage{amssymb,amsfonts,amsmath,amsthm} % математические дополнения от АМС
    \usepackage{indentfirst} % отделять первую строку раздела абзацным отступом тоже
% Поля
    \usepackage{geometry}
    \geometry{left=3cm}
    \geometry{right=1.5cm}
    \geometry{top=2.4cm}
    \geometry{bottom=2.cm}
    \usepackage{floatflt}

%%%%%%%%%%%%%%%%%%%%%%%%%%%%%%%     

    \linespread{1.5} % полуторный интервал
    \renewcommand{\rmdefault}{ftm} % Times New Roman
    \frenchspacing





\begin{document}

\begin{titlepage}

\begin{center}
САНКТ-ПЕТЕРБУРГСКИЙ ПОЛИТЕХНИЧЕСКИЙ УНИВЕРСИТЕТ\\ ПЕТРА ВЕЛИКОГО\\
\vspace{0.1cm}
Институт Энергетики\\
\hrulefill \\
\vspace{0.5cm}
Кафедра "Теплофизика энергетических установок"\\

\end{center}

\vspace{5cm}
\begin{center}
\begin{large}
Лабораторная работа \\
"Структура потока в циклонно-вихревой камере"
\end{large}
\end{center}

\vspace{5cm}
\hspace{5cm} Студент гр. 3231303/80005 \hrulefill Козлова А.М.

\hspace{5cm} Студент гр. 3231303/80005 \hrulefill Григорьева Д.И.

\vspace{0.5cm}
\hspace{5cm} Преподаватель \hrulefill Коршунов А.В. \\


\vfill
\begin{center}
Санкт-Петербург\\
2020
\end{center}


\end{titlepage}



\tableofcontents
\newpage

\section{Введение}

Циклонная камера – наиболее распространенное устройство для
организации вращающихся потоков. Вращающиеся течения широко
используются для интенсификации тепломассообмена в вихревых горелках, циклонных топках и камерах сгорания, нагревательных и плавильных устройствах, в рециркуляционных сушилках, а также в сепараторах и гидроциклонах для очистки и разделения веществ и материалов.

Между зонами квазитвердого и квазипотенциального вращения существует довольно большая область вихревого
течения, в которой по мере удаления от ядра вихря завихренность
снижается до весьма малых значений, соответствующих квазипотенциальному вращению. Реальную структуру плоского вихревого потока можно изучить двумя способами: либо численным моделированием течения вязкой жидкости, либо экспериментально.

В данной работе избран экспериментальный путь, а в качестве
объекта исследования – циклонно-вихревая камера, в среднем по высоте сечения которой поток можно считать приблизительно плоским:
в этом сечении продольная (в направлении оси камеры) составляющая скорости невелика, а радиальная составляющая пренебрежимо
мала. Кроме изучения структуры вращательного движения газа в
среднем сечении циклонно-вихревой камеры, измеряют также ее гидравлическое сопротивление и некоторые другие характеристики.

\section{Описание установки}
Циклонно-вихревая камера изображена на рис. 1. Внутренний диаметр камеры $D = 140$ мм, высота $L = 220$ мм, диаметр выходного отверстия $d_{\text{вых}} = 50$ мм. Воздух от вентилятора 1 поступает в
камеру через две продольные щели, формирующие тангенциальное
течение. Площадь живого сечения обеих щелей составляет 400 мм$^2$.
Для измерения расхода воздуха установлена диафрагма 2 с манометром 1 (номера манометров на схеме обозначены цифрами в кружках). Для измерения температуры воздуха служит термометр.
Для измерения тангенциальной составляющей скорости используется цилиндрический зонд (рис. 2), который представляет
собой тонкий (диаметром 3 мм) цилиндр 1. На поверхности цилиндра
выполнены три отверстия 2. С помощью тонких трубок, проложенных
внутри зонда, эти отверстия через штуцер 3 соединены с манометрами 4, 5 и 6 эластичными шлангами. 
\begin{figure}[h]\center
\includegraphics[width=90mm]{1.png}
\caption{Схема установки (а) и среднее по высоте сечение (б)} \label{fig:1}
\end{figure}

\begin{figure}[h!]\center
\includegraphics[width=90mm]{2.png}
\caption{Цилиндрический зонд} \label{fig:2}
\end{figure}










\section{Результаты эксперимента}

\begin{quote}
\flushright Таблица 1.
\end{quote}
\begin{center}
Результаты измерений\\
\vspace{0.1cm}
\includegraphics[width=160mm]{3.png}
\end{center}

\begin{quote}
\flushright Таблица 2.
\end{quote}
\begin{center}
Результаты измерений\\
\vspace{0.1cm}
\includegraphics[width=160mm]{4.png}
\end{center}



\section{Обработка результатов}

Плотность воздуха рассчитывается по уравнению состояния идеального газа
\begin{equation}
\rho_{\text{возд}} = \dfrac{p_{\text{б}}}{R_{\text{возд}}T_{\text{возд}}}, \text{ кг/м$^3$},
\end{equation}
где $p_{\text{б}}$ – атмосферное давление, Па; $R_{\text{возд}} = 287 \text{ Дж/(кг$\cdot$К)}$ – газовая постоянная воздуха; $T_{\text{возд}} = t + 273$ К – средняя абсолютная температура
воздуха в камере.

Рассчитываются поля скорости и давления в циклонно-вихревой
камере.

Полное давление потока 
\begin{equation}
p_{\text{полн}} = 9,81\text{К$_{\text{ц}}h_5$}, \text{ Па}.
\end{equation}

Статическое давление
\begin{equation}
p_{\text{ст}}=p_{\text{полн}} - \dfrac{9,81h_6}{\text{К$_{\text{ц}}$ - К$_{\text{б}}$}}, \text{ Па}.
\end{equation}


Тангенциальная составляющая скорости
\begin{equation}
w_{\tau} = \sqrt{\dfrac{2\cdot9,81\ h_6}{\rho\left(\text{К$_{\text{ц}}$-К$_{\text{б}}$} \right)}}\cos\varphi, \text{ м/с},
\end{equation}
где $\varphi$ – угол между горизонтальной плоскостью и направлением вектора скорости.

По результатам расчетов строятся графики $w_{\tau}=f_1(r),\ p_{\text{п}}=f_2(r),\ p_{\text{ст}}=f_3(r).$

В таком же виде обрабатываются данные измерений статического
давления на торцах камеры $p_{7\dots 16} = f_1(r)$ и $p_{17\dots 22} = f_2(r)$.
Здесь $p_{7\dots 16} =9,81h_{7\dots 16}$, $p_{17\dots 22} =9,81h_{17\dots 22}$
при условии, что $h_{7\dots 16}$ и $h_{17\dots 22}$
измерены в мм. вод. ст.

Окончательные результаты нужно представить в безразмерном виде. В качестве масштаба скорости используют максимальное значение $w_{\tau max}$, а в качестве масштаба давлений – динамический напор $\rho w_{\tau max}^2/2$. 
За масштаб длины принимают $r_{max}$ – радиус, на котором
тангенциальная составляющая скорости достигает максимума
$(w_{\tau}=w_{\tau max})$.
Тогда безразмерный радиус $\eta=r/r_{max}$.


Строятся графики безразмерных функций аргумента $\eta$:

$\bar w_{\tau} = \dfrac{w_{\tau}}{w_{\tau max}},\ \bar p_{\text{ст}} = \dfrac{p_{\text{ст}}}{\rho w_{\tau max}^2/2},\ \bar p_{\text{пол}}=\dfrac{p_{\text{пол}}}{\rho w_{\tau max}^2/2},\ \bar p_{7\dots 16} = \dfrac{p_{7\dots 16}}{\rho w_{\tau max}^2/2} $ и $\bar p_{17\dots 22} =\\= \dfrac{p_{17\dots 22}}{\rho w_{\tau max}^2/2}$.
\\

Находится функция $\bar w_{\tau} (\eta)$, аппроксимирующая измеренный профиль скорости. Для аппроксимации выбирается функция, предложенную П.М. Михайловым[Деветерикова М.А., Михайлов П.М. О новой аппроксимации для тангенциальной скорости при расчете аэродинамических характериситк циклонно-вихревых камер // Информ. Обеспечение, адаптация, динамика и прочность систем – 74.
Куйбышев, 1976. С. 395–399.]:
\begin{equation}
\bar w_{\tau_\text{расч}} = \dfrac{1}{\sqrt{1+1/b^2\left(\eta - 1/\eta \right)^2}},
\end{equation}
где $b$ – постоянная, определяемая из эксперимента.

 Расстояние между точками пересечения кривой $\bar w_{\tau}(\eta)$ с горизонталью $\bar w = \sqrt{0,5} \approx $0,707
равно искомой величине $b$. 

В зоне квазитвердого вращения завихренность плоского осесимметричного течения
должна быть примерно постоянной. В переходной области она должна снижаться, приближаясь к нулю в зоне потенциального вращения. 

В зоне квазитвердого вращения
циркуляция скорости должна увеличиваться примерно пропорционально квадрату радиуса, затем в переходной области темп ее роста
должен снижаться, а в квазипотенциальной зоне она должна сохраняться приблизительно постоянной.

Для проверки, насколько опытные данные отвечают описанным особенностям течения, строятся безразмерные завихренность
и циркуляция скорости в зависимости от безразмерного радиуса с использованием аппроксимирующей профиль скорости функции.

В этом случае расчетные формулы примут вид
\begin{equation}
\bar \omega_{\text{расч}} = \dfrac{1}{2\eta}\dfrac{\partial \left(\bar w_{\tau\text{расч}} \eta \right)}{\partial \eta} = \dfrac{\bar w_{\tau\text{расч}}}{2\eta}\left[1+\dfrac{1-\eta^4}{\left(\eta^2 -1 \right) + b^2\eta^2} \right],
\end{equation}
\begin{equation}
\bar {\text{Г}}_{\text{расч}} = 2\pi \bar w_{\tau\text{расч}} \eta = \dfrac{2\pi\eta}{\sqrt{1+\dfrac{1}{b^2}}\left(\eta - \dfrac{1}{\eta} \right)^2}.
\end{equation}

На график $\bar {\text{Г}}_{\text{расч}}(\eta)$
наносятся данные опыта, обработанные по
формуле
\begin{equation}
\bar {\text{Г}}_{\text{эксп}} = 2\pi \bar w_{\tau\text{эксп}} \eta .
\end{equation}

В заключение определяется гидравлическое сопротивления циклонно-вихревой камеры. Для этого предварительно необходимо вычислить среднюю (по расходу) скорость воздуха на входе.

Расход воздуха рассчитывается по перепаду давлений на дроссельной
шайбе:
\begin{equation}
q_v = \alpha F_{\text{д.ш.}}\sqrt{2 \cdot 9,81 \dfrac{p_1}{\rho}}, \text{ м$^3$/с,}
\end{equation}
где $\alpha$ = 0,8 – коэффициент расхода дроссельной шайбы;
$F_{\text{д.ш.}} = \pi d_{\text{д.ш.}}/4$ , м$^2$,– площадь отверстия в дроссельной шайбе
($d_{\text{д.ш.}} = 14$ мм); $p_1$, мм вод. ст. – показания манометра 1.

Средняя скорость воздуха на входе в камеру
\begin{equation}
w_{\text{вх}} = \dfrac{q_v}{\sum F_{\text{вх}}}, \text{ м/с,}
\end{equation}
где $F_{\text{вх}} = 400$ мм$^2$ – суммарная площадь входного сечения.

Общее гидравлическое сопротивление камеры
\begin{equation}
\Delta p_{\text{к}}=p_{\text{вх}}  +\rho w^2_{\text{вх}}/2, \text{ Па,}
\end{equation}
где $p_{\text{вх}} = 9,81\cdot h_2$ , Па, – давление на входе в камеру;
$h_2$– показания
манометра 2,\\ мм вод. ст.

Результаты расчетов приведены в таблице 3. Ниже приводятся примеры выполнения расчетов.

\begin{quote}
\flushright Таблица 3.
\end{quote}
\begin{center}
Результаты расчетов\\
\vspace{0.1cm}
\includegraphics[width=180mm]{5.png}
\end{center}


Температура воздуха в начале опыта $t_1 = 24^\circ C.$

Температура воздуха в конце опыта $t_2 = 28^\circ C.$

Атмосферное давление $p_{\text{б}} = 742$ мм.рт.ст.
\begin{spacing}{2.5}
$\rho_{\text{возд}} = \dfrac{742 \cdot 133,3 \text{ Па}}{287\dfrac{\text{Дж}}{\text{кг$\cdot$К}}\cdot(273+26)\text{К}} \approx 1,15 \text{ кг/м$^3$}$

$p_{\text{полн}} = 9,81 \cdot 0,97 \cdot 132\ \text{мм вод. ст.} \approx 1275,104 \text{ Па}$

$p_{\text{ст}} =1275,104 \text{ Па} - \dfrac{ 9,81\cdot 31\ \text{мм вод. ст.}}{0,887} \approx 932,25 \text{ Па}$

$w_{\tau} = \sqrt{\dfrac{2\cdot 9,81 \cdot 31\ \text{мм вод. ст.} }{1,15\ \dfrac{\text{кг}}{\text{м$^3$}}\cdot 0,887}}\cos \left(\dfrac{7^\circ}{180^\circ}\pi \right) \approx 20,24 \text{ м/с}$

Из рис. 8 (см. приложение) видно, что $b \approx 1,4$

$r_{max} = 21$ мм, тогда $\eta = \dfrac{68\text{ мм}}{21 \text{ мм}} \approx 3,24$

$\bar w_{\tau\text{расч}} = \dfrac{1}{\sqrt{1+\dfrac{1}{1,4^2}\left(3,24 - \dfrac{1}{3,24}\right)^2}} \approx 0,43$
\\

$\bar \omega_{\text{расч}}= \dfrac{0,43}{2\cdot 3,24 }\left[1+\dfrac{1-3,24^4}{\left(3,24^2-1\right)^2 + 1,4^2\cdot3,24^2} \right] \approx -0,0009$
\end{spacing}
\begin{spacing}{2.0}
$\bar {\text{Г}}_{\text{расч}}= \dfrac{2\pi \cdot 3,24 }{\sqrt{1+\dfrac{1}{1,4^2}\left(3,24 - {\dfrac{1}{3,24}}\right)^2}} \approx 8,773$
\end{spacing}

\begin{spacing}{2.5}
$\bar {\text{Г}}_{\text{эксп}}= 2\pi \cdot 0,55 \cdot 3,24  \approx 11,28$

$q_v=0,8\cdot \dfrac{\pi \left(14 \cdot 10^{-3} \text{ м}\right)^2}{4} \cdot \sqrt{2\cdot 9,81 \cdot\dfrac{177\text{ мм вод. ст.}}{1,15 \text{ кг/м$^3$}}} \approx 6,76\cdot10^{-3}\ \dfrac{\text{м}^3}{\text{с}} $

$w_{\text{вх}} = \dfrac{6,76\cdot10^{-3}\ \dfrac{\text{м}^3}{\text{с}}}{400\cdot 10^{-6}\ \text{м}^2} \approx 16,91\ \dfrac{\text{м}}{\text{с}} $
\end{spacing}
$\Delta p_{\text{к}} = 9,81 \cdot 43,5\ \text{мм вод. ст.} + \dfrac{1,15\ \dfrac{\text{кг}}{\text{м}^3} \cdot \left(16,91\ \dfrac{\text{м}}{\text{с}}\right)^2}{2} \approx 591,35 \text{ Па}$

Ниже приводятся графики, построенные в ходе обработки результатов.

\begin{figure}[h]\center
\includegraphics[width=90mm]{6.png}
\caption{График зависимости $w_{\tau}$=$f_1$(r)} \label{fig:3}
\end{figure}

\begin{figure}[h]\center
\includegraphics[width=90mm]{7.png}
\caption{График зависимости $p_{\text{полн}}$=$f_2$(r)} \label{fig:3}
\end{figure}

\begin{figure}[h]\center
\includegraphics[width=90mm]{8.png}
\caption{График зависимости $p_{\text{ст}}$=$f_3$(r)} \label{fig:3}
\end{figure}

\begin{figure}[h]\center
\includegraphics[width=90mm]{9.png}
\caption{График зависимости $p_{\text{7...16}}$=$f_4$(r)} \label{fig:3}
\end{figure}

\begin{figure}[h]\center
\includegraphics[width=90mm]{10.png}
\caption{График зависимости $p_{\text{17...22}}$=$f_5$(r)} \label{fig:3}
\end{figure}

\begin{figure}[h]\center
\includegraphics[width=90mm]{11.png}
\caption{График зависимости $\bar w_{\tau}=f_1(\eta) $}\label{fig:3}
\end{figure}

\begin{figure}[h]\center
\includegraphics[width=90mm]{12.png}
\caption{График зависимости $\bar p_{\text{ст}}=f_2(\eta) $}\label{fig:3}
\end{figure}

\begin{figure}[h]\center
\includegraphics[width=90mm]{13.png}
\caption{График зависимости $\bar p_{\text{полн}}=f_3(\eta) $}\label{fig:3}
\end{figure}

\begin{figure}[h]\center
\includegraphics[width=90mm]{14.png}
\caption{График зависимости $\bar p_{\text{7...16}}=f_4(\eta) $}\label{fig:3}
\end{figure}

\begin{figure}[h]\center
\includegraphics[width=90mm]{15.png}
\caption{График зависимости $\bar p_{\text{17...22}}=f_5(\eta) $}\label{fig:3}
\end{figure}

\begin{figure}[h]\center
\includegraphics[width=90mm]{16.png}
\caption{График зависимости $\bar {\text{Г}}_{\text{эксп}}=f_6(\eta) $, наложенный на график зависимости $\bar {\text{Г}}_{\text{расч}}=f_7(\eta)$}\label{fig:3}
\end{figure}

\begin{figure}[h]\center
\includegraphics[width=90mm]{17.png}
\caption{График зависимости $\bar \omega_{\text{расч}}=f_8(\eta) $}\label{fig:3}
\end{figure}

\section{Выводы}
В результате проведённой работы была изучена реальная структура плоского вихревого потока, было экспериментально подтверждено, что он имеет три зоны: зону квазитвердого вращения, переходную область и зону квазипотенциального вращения.
В зоне квазитвердого вращения завихрённость должна быть почти постоянна (так как скорость в этой области прямо пропорциональна радиусу), в переходной области она должна снижаться, а в квазипотенциальной зоне приближаться к нулю из-за отсутствия вихревого движения. Опытные данные отвечают описанным особянностям течения для завихренности течения (см. рис. 15). Что касается циркуляции скрости, то в зоне квазитвердого вращения она должна увеличиваться примерно пропорционально квадрату радиуса, затем в переходной области темп ее роста
должен снижаться, а в квазипотенциальной зоне она должна сохраняться приблизительно постоянной. Экспериментальный график циркуляции скорости(рис. 14) не точно совпадает с расчетными значениями, что видно при его сравнении с рис. 13, сравнительное совпадение имеется лишь для переходной области.



\end{document}