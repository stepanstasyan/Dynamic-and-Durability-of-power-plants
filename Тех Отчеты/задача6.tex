\documentclass[12pt, a4paper]{article}

\usepackage{lipsum}% http://ctan.org/pkg/lipsum
\setlength{\parindent}{1.25cm}

\usepackage{hyperref}
\hypersetup{
    colorlinks=true, % make the links colored
    linkcolor=black, % color TOC links in blue
    urlcolor=red, % color URLs in red
    linktoc=all % 'all' will create links for everything in the TOC
}
\usepackage{mathtext}
\usepackage[T2A]{fontenc}  % поддержка кириллицы в ЛаТеХ
\usepackage[utf8]{inputenc} % кодировка
\usepackage[english, russian]{babel} % определение языков в документе
% \usepackage{pscyr} % красивые кириллические шриф ты
% \usepackage{mathspec}
% \usepackage{upgreek} % добавляет прямые греческие символы
\usepackage[eulergreek, italic]{mathastext}
% \usepackage{lscape} % для включения альбомных страниц (широкие таблицы, графики и т.д.)

% \usepackage{nath}

\usepackage{amsmath} % многострочные формулы
% \usepackage{breqn}  % автоматические многострочные формулы
\usepackage{amstext} % определяет \textit{} для включения в формулы текста
\usepackage{amssymb} % набор символов
% \usepackage{makecell} % работа с ячейками в таблицах
% \usepackage[math-style=upright]{unicode-math}
\usepackage{indentfirst}  % делает отступ в начале параграфа

% \usepackage{lmodern}
% \usepackage{listings} % для листингов программ
\usepackage{pdflscape} % альбомные страницы

\usepackage{longtable} % для многостраничных таблиц
\usepackage{multirow}  % для объединения ячеек таблиц
\usepackage{multicol}  % для объединения ячеек таблиц

\usepackage{mathtools}
% \usepackage{makeidx}
% \usepackage{amssymb}
% \usepackage{amsfonts}
\usepackage{cite}
% \usepackage{hyperref}

\usepackage{enumerate} % нумерация списков
\usepackage{enumitem}

% \usepackage{txfonts}
% \usepackage{kpfonts}

% \usepackage{float}
% \usepackage{mathdots}
% \usepackage{pdflscape}
\usepackage{array}

\pagestyle{empty}

% \usepackage{ltxtable}
% \usepackage{lipsum}

\usepackage{geometry} % размеры листа
\geometry{left = 3cm} % размеры листа
\geometry{right = 2cm} % размеры листа
\geometry{top = 2cm} % размеры листа
\geometry{bottom = 2cm} % размеры листа

\usepackage[figurename=Рисунок]{caption}
\usepackage{subcaption}

\DeclareCaptionLabelFormat{continued}{Продолжение таблицы~#2}
\DeclareCaptionLabelFormat{gostfigure}{Рисунок #2}
\DeclareCaptionLabelFormat{gosttable}{Таблица #2}
\DeclareCaptionLabelSeparator{gost}{~---~}

\captionsetup[figure]{justification = centering}
\captionsetup{labelsep=gost}
\captionsetup*[figure]{labelformat=gostfigure}

\usepackage{graphicx}
% \setlength\extrarowheight{6pt}

\renewcommand{\rmdefault}{ftm} % переключение на общий шрифт документа Times New Roman (пакет pscyr)
%\renewcommand\theadfont{\normalsize}

\linespread{1.3}

\frenchspacing

% \usepackage{tikz}
% \usetikzlibrary{shapes.geometric, arrows}
% \usetikzlibrary{automata,positioning}


\usepackage{titlesec}

% Настройка формата разделов и подразделов
\titleformat{\section}{\normalfont\normalsize\bfseries}{\thesection}{1em}{}
\titleformat{\subsection}{\normalfont\normalsize\bfseries}{\thesubsection}{1em}{}
\titleformat{\subsubsection}{\normalfont\normalsize\bfseries}{\thesubsubsection}{1em}{}

% Настройка расстояния до разделов и подразделов и после
\titlespacing*{\section}{\parindent}{0pt}{18pt}
\titlespacing*{\subsection}{\parindent}{18pt}{18pt}
\titlespacing*{\subsubsection}{\parindent}{18pt}{18pt}

% Каждый раздел с новой страницы
\newcommand{\sectionbreak}{\clearpage}

% \addto\captionsenglish{
% \renewcommand\contentsname{СОДЕРЖАНИЕ}
% }
\addto\captionsrussian{
    \renewcommand\contentsname{\hfill СОДЕРЖАНИЕ\hfill}
}

%\usepackage{accsupp}

\makeatletter
\renewcommand*\l@section{\@dottedtocline{1}{1.5em}{2.3em}}
\renewcommand*\l@subsection{\@dottedtocline{1}{1.5em}{2.3em}}
\renewcommand*\l@subsubsection{\@dottedtocline{1}{1.5em}{2.3em}}
% \newcommand\cdot {\copyable}{%
%   \begingroup
%   \@sanitize
%   \catcode`\%=14 % allow % as comment char, also needed for \%
%   \@copyable
% }
% \newcommand\cdot {\@copyable}[1]{%
%   \endgroup
%   \BeginAccSupp{%
%     ActualText=\detokenize{#1},%
%     method=escape,
%   }%
%   \scantokens{#1}%
%   \EndAccSupp{}%
% }
\makeatother



\begin{document}

\begin{titlepage}

\begin{center}
Министерство образования и науки Российской Федерации\\
Федеральное государственное автономного образовательное учреждение высшего образования\\
\hrulefill\\
\vspace{0.5cm}
САНКТ-ПЕТЕРБУРГСКИЙ ПОЛИТЕХНИЧЕСКИЙ УНИВЕРСИТЕТ\\ ПЕТРА ВЕЛИКОГО\\
\vspace{0.5cm}
Институт Энергетики\\
Высшая школа энергетического машиностроения\\

\end{center}

\vspace{5cm}
\begin{center}
\begin{large}
Лабораторная работа №6\\
"Определение собственных частот поперечных колебаний рабочей лопатки, расположенной на вращающемся диске"
\end{large}
\end{center}

\vspace{5cm}
\hspace{5cm} Студент гр. 3231303/81001 \hrulefill Степанов С.С.


\vspace{0.5cm}
\hspace{5cm} Преподаватель \hrulefill Курнухин А.А. \\


\vfill
\begin{center}
Санкт-Петербург\\
2021
\end{center}


\end{titlepage}

\tableofcontents
\newpage

\section{Исходные данные}

Растояние между лопатками, [м]
\[l = 1,24.\]

Ширина профиля, [мм]
\[b = 122.\]

Угол установки, [град]
\[\beta_{y} = 90^\circ.\]

Толщина профиля, [мм]
\[h = 36.\]

Плотность стали, [кг/м\textsuperscript{3}]
\[\rho = 7800.\]

Модуль упругости, [МПа]
\[Е = 20000.\]

Какая-то дельта, [мм]
\[\delta = 32.\]

Какие то коэффициенты
\[k_{1}l = 1,98,\]

\[k_{2}l = 4,674,\]

\[k_{3}l = 7,91.\]

Отношение диаметра к длине
\[\frac{d}{l} = 3,7.\]

Частота вращения, [Гц]
\[n = 50.\]


\section{Определение собственной частоты поперечных колебаний рабочей лопатки,
  расположенной на вращающемся диске}


Эмпирические формулы:

\[F = 0,69\cdot b\cdot\delta = 0,69\cdot122\cdot32 = 2693,8\ {мм}^{2},\]

\[Y_{\min} = 0,041\cdot b\cdot\delta\cdot\left( h^{2} + \delta^{2} \right) = 0,041\cdot122\cdot32\left( 36^{2} + 32^{2} \right) = 371348\ {мм}^{4}.\ \]

Собственные частоты, [Гц]

\[p_{k} = \left( b_{k}\cdot l \right)^{2}\cdot\sqrt{\frac{{E\cdot Y}}{\rho\cdot F\cdot l^{4}}} ;\ \]

\[p_{1} = \left( 1,98 \right)^{2}\cdot\sqrt{\frac{2\cdot10^{11}\cdot371348\cdot10^{- 6}}{7800\cdot2693,8\cdot{1,24}^{4}}} = 151,59;\ \]

\[p_{2} = 844,73;\ \]

\[p_{3} = 2419,32.\ \]

Поправка на ужесточение лопатки:

\[\ B = 0,786\left( \frac{d}{l} \right) + 0,407 - \cos^{2}\beta_{y} = 0,786\cdot3,7 + 0,407 - \cos^{2}90 = 3,315,\]

Собственные частоты с учетом поправки, [Гц]
\[p_{1\omega} = \sqrt{p_{1}^{2} + B\cdot n^{2}} = \sqrt{{151,59}^{2} + 3,315{\cdot50}^{2}} = 176,8;\ \]

\[p_{2\omega} = 849,6;\ \]

\[p_{3\omega}= 2421,3.\ \]

\section{Данные для построения вибрационной диаграммы}

\[p_{\text{k}\omega} = \sqrt{p_{k}^{2} + Bn^{2}}\]

\[p = k\frac{n}{60}\]\\

\begin{longtable}{ccccccccc}
& & & k & & & & &\\
n & \(p_{1\omega}\) & \(p_{2\omega}\) & 1 & 2 & 3 & 4 & 5 & 6\\
0 & 151,59 & 844,73 & 0 & 0 & 0 & 0 & 0 & 0\\
500 & 152,34 & 844,87 & 8,33 & 16,67 & 25 & 33,33 & 41,67 & 50\\
1000 & 154,61 & 845,28 & 16,67 & 33,33 & 50 & 66,67 & 83,33 & 100\\
1500 & 158,28 & 845,96 & 25 & 50 & 75 & 100 & 125 & 150\\
2000 & 163,27 & 846,9 & 33,33 & 66,67 & 100 & 133,33 & 166,67 & 200\\
2500 & 169,54 & 848,14 & 41,67 & 83,33 & 125 & 166,67 & 208,33 & 250\\
3000 & 176,82 & 849,62 & 50 & 100 & 150 & 200 & 250 & 300\\
3500 & 185,06 & 851,37 & 58,33 & 116,67 & 175 & 233,33 & 291,67 & 350\\
4000 & 194,24 & 853,41 & 66,67 & 133,33 & 200 & 266,67 & 333,33 & 400\\
4500 & 204,03 & 855,7 & 75 & 150 & 225 & 300 & 375 & 450\\
5000 & 214,43 & 858,24 & 83,33 & 166,67 & 250 & 333,33 & 416,67 & 500\\
6000 & 236,92 & 864,13 & 100 & 200 & 300 & 400 & 500 & 600\\
7000 & 261,01 & 871,04 & 116,67 & 233,33 & 350 & 466,67 & 583,33 & 700\\
8000 & 286,15 & 878,9 & 133,33 & 266,67 & 400 & 533,33 & 666,67 &800\\
9000 & 312,36 & 887,78 & 150 & 300 & 450 & 600 & 750 & 900\\
10000 & 339,26 & 897,6 & 166,67 & 333,33 & 500 & 666,67 & 833,33 & 1000\\
\end{longtable}

\section{Вибрационная диаграмма}

Вывод: лопатка вибронадёжна.

\end{document}
