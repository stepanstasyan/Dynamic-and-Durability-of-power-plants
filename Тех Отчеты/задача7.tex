\documentclass[12pt, a4paper]{article}

\usepackage{lipsum}% http://ctan.org/pkg/lipsum
\setlength{\parindent}{1.25cm}

\usepackage{hyperref}
\hypersetup{
    colorlinks=true, % make the links colored
    linkcolor=black, % color TOC links in blue
    urlcolor=red, % color URLs in red
    linktoc=all % 'all' will create links for everything in the TOC
}
\usepackage{mathtext}
\usepackage[T2A]{fontenc}  % поддержка кириллицы в ЛаТеХ
\usepackage[utf8]{inputenc} % кодировка
\usepackage[english, russian]{babel} % определение языков в документе
% \usepackage{pscyr} % красивые кириллические шриф ты
% \usepackage{mathspec}
% \usepackage{upgreek} % добавляет прямые греческие символы
\usepackage[eulergreek, italic]{mathastext}
% \usepackage{lscape} % для включения альбомных страниц (широкие таблицы, графики и т.д.)

% \usepackage{nath}

\usepackage{amsmath} % многострочные формулы
% \usepackage{breqn}  % автоматические многострочные формулы
\usepackage{amstext} % определяет \textit{} для включения в формулы текста
\usepackage{amssymb} % набор символов
% \usepackage{makecell} % работа с ячейками в таблицах
% \usepackage[math-style=upright]{unicode-math}
\usepackage{indentfirst}  % делает отступ в начале параграфа

% \usepackage{lmodern}
% \usepackage{listings} % для листингов программ
\usepackage{pdflscape} % альбомные страницы

\usepackage{longtable} % для многостраничных таблиц
\usepackage{multirow}  % для объединения ячеек таблиц
\usepackage{multicol}  % для объединения ячеек таблиц

\usepackage{mathtools}
% \usepackage{makeidx}
% \usepackage{amssymb}
% \usepackage{amsfonts}
\usepackage{cite}
% \usepackage{hyperref}

\usepackage{enumerate} % нумерация списков
\usepackage{enumitem}

% \usepackage{txfonts}
% \usepackage{kpfonts}

% \usepackage{float}
% \usepackage{mathdots}
% \usepackage{pdflscape}
\usepackage{array}

\pagestyle{empty}

% \usepackage{ltxtable}
% \usepackage{lipsum}

\usepackage{geometry} % размеры листа
\geometry{left = 3cm} % размеры листа
\geometry{right = 2cm} % размеры листа
\geometry{top = 2cm} % размеры листа
\geometry{bottom = 2cm} % размеры листа

\usepackage[figurename=Рисунок]{caption}
\usepackage{subcaption}

\DeclareCaptionLabelFormat{continued}{Продолжение таблицы~#2}
\DeclareCaptionLabelFormat{gostfigure}{Рисунок #2}
\DeclareCaptionLabelFormat{gosttable}{Таблица #2}
\DeclareCaptionLabelSeparator{gost}{~---~}

\captionsetup[figure]{justification = centering}
\captionsetup{labelsep=gost}
\captionsetup*[figure]{labelformat=gostfigure}

\usepackage{graphicx}
% \setlength\extrarowheight{6pt}

\renewcommand{\rmdefault}{ftm} % переключение на общий шрифт документа Times New Roman (пакет pscyr)
%\renewcommand\theadfont{\normalsize}

\linespread{1.3}

\frenchspacing

% \usepackage{tikz}
% \usetikzlibrary{shapes.geometric, arrows}
% \usetikzlibrary{automata,positioning}


\usepackage{titlesec}

% Настройка формата разделов и подразделов
\titleformat{\section}{\normalfont\normalsize\bfseries}{\thesection}{1em}{}
\titleformat{\subsection}{\normalfont\normalsize\bfseries}{\thesubsection}{1em}{}
\titleformat{\subsubsection}{\normalfont\normalsize\bfseries}{\thesubsubsection}{1em}{}

% Настройка расстояния до разделов и подразделов и после
\titlespacing*{\section}{\parindent}{0pt}{18pt}
\titlespacing*{\subsection}{\parindent}{18pt}{18pt}
\titlespacing*{\subsubsection}{\parindent}{18pt}{18pt}

% Каждый раздел с новой страницы
\newcommand{\sectionbreak}{\clearpage}

% \addto\captionsenglish{
% \renewcommand\contentsname{СОДЕРЖАНИЕ}
% }
\addto\captionsrussian{
    \renewcommand\contentsname{\hfill СОДЕРЖАНИЕ\hfill}
}

%\usepackage{accsupp}

\makeatletter
\renewcommand*\l@section{\@dottedtocline{1}{1.5em}{2.3em}}
\renewcommand*\l@subsection{\@dottedtocline{1}{1.5em}{2.3em}}
\renewcommand*\l@subsubsection{\@dottedtocline{1}{1.5em}{2.3em}}
% \newcommand\cdot {\copyable}{%
%   \begingroup
%   \@sanitize
%   \catcode`\%=14 % allow % as comment char, also needed for \%
%   \@copyable
% }
% \newcommand\cdot {\@copyable}[1]{%
%   \endgroup
%   \BeginAccSupp{%
%     ActualText=\detokenize{#1},%
%     method=escape,
%   }%
%   \scantokens{#1}%
%   \EndAccSupp{}%
% }
\makeatother



\begin{document}

\begin{titlepage}

\begin{center}
Министерство образования и науки Российской Федерации\\
Федеральное государственное автономного образовательное учреждение высшего образования\\
\hrulefill\\
\vspace{0.5cm}
САНКТ-ПЕТЕРБУРГСКИЙ ПОЛИТЕХНИЧЕСКИЙ УНИВЕРСИТЕТ\\ ПЕТРА ВЕЛИКОГО\\
\vspace{0.5cm}
Институт Энергетики\\
Высшая школа энергетического машиностроения\\

\end{center}

\vspace{5cm}
\begin{center}
\begin{large}
Лабораторная работа №2\\
"Определение частот собственных колебаний вращающегося диска"
\end{large}
\end{center}

\vspace{5cm}
\hspace{5cm} Студент гр. 3231303/81001 \hrulefill Степанов С.С.


\vspace{0.5cm}
\hspace{5cm} Преподаватель \hrulefill Курнухин А.А. \\


\vfill
\begin{center}
Санкт-Петербург\\
2021
\end{center}


\end{titlepage}

\tableofcontents
\newpage

\section{Исходные данные}

\(l = 1,24\ м;\)

\(b = 122\ мм;\)

\(\delta = 32\ мм;\) \(\ \)

\[h = 36\ мм;\]

\[\rho = 7800\ \frac{кг}{м^{3}};\]

\[Е = 2*10^{11}\ Па\]

n = 3000 об/мин = 50 Гц

\[\frac{d}{l} = 3,7\]

1) Определить частоту крутильных колебаний рабочей лопатки

\[\frac{\partial}{\partial x}\left\lbrack \text{Gk}\frac{\partial\varphi}{\partial x} \right\rbrack - \rho Y_{p}\frac{\partial^{2}\varphi}{\partial\tau^{2}} = 0\]

\[\varphi_{n}\left( x,\tau \right) = \Phi_{n}(x)T_{n}\left( \tau \right)\]

\[Gk\cdot\Phi^{''}T - \rho Y_{p}\Phi\ddot{T} = 0\]

\[\frac{\text{Gk}}{\rho Y_{p}}\frac{\Phi''}{\Phi} = \frac{\ddot{T}}{T} = - \lambda^{2}\]

\[T = N_{\varphi}cos(\lambda\tau - \alpha)\]

\[\Phi^{''} + k^{2}\Phi = 0,\ где\ k^{2} = \frac{\lambda^{2}\rho Y_{p}}{\text{Gk}}\]

\[\Phi\left( x \right) = C_{1}\sin{kx + C_{2}\cos\text{kx}}\]

Граничные условия:

\[x = 0 \Rightarrow \varphi = 0;\ \Phi\left( 0 \right) = 0 \Rightarrow C_{2} = 0\]

\[x = l,\ M = Gk\frac{\partial\varphi}{\partial x} = GkT\Phi^{'} = 0;\ \Phi^{'}\left( l \right) = 0 \Rightarrow C_{1}\cos{kl = 0}\]

\[C_{1} \neq 0 \Rightarrow kl = 0\]

\(k_{n}l = (2n - 1)\frac{\pi}{2}\) при n\(\  \in N\)

\[\lambda_{n} = k_{n}\sqrt{\frac{\text{GK}}{\rho Y_{p}}}\]

\[f_{1кр} = \frac{\lambda_{1}}{2\pi} = \frac{1}{4l}\sqrt{\frac{\text{GK}}{\rho Y_{p}}}\]

\[k = \frac{0,162b\delta^{3}}{1 + 1,43\left( \frac{\delta}{b} \right)^{2} + 2,87\left( \frac{h}{b} \right)^{2}} = \frac{0,162*122{*32}^{3}}{1 + 1,43\left( \frac{32}{122} \right)^{2} + 2,87\left( \frac{36}{122} \right)^{2}} = 0,48*10^{- 6}\ м^{4}\]

\[Y_{p} = 0,038\delta b^{3} + 0,04b\delta\left( \delta^{2} + h^{2} \right) = 0,038*32*122^{3} + 0,04*122*32*\left( 32^{2} + 36^{2} \right) = 2,57*10^{- 6}\ м^{4}\]

\[f_{1кр} = \frac{1}{4*1,24}\sqrt{\frac{7,93*10^{10}*0,48*10^{- 6}}{7800*2,57*10^{- 6}}} = 275,6\ Гц\]

\[f_{2кр} = 3f_{1кр} = 3*275,6 = 826,8\ Гц\]

\[f_{3кр} = 5f_{1кр} = 5*275,6 = 1377,98\ Гц\]

\[\Phi_{n}(x) = C_{1n}\sin k_{n}x\]

\[C_{1n} = \sqrt{\frac{2}{\rho Y_{p}l}} = \sqrt{\frac{2}{7800*2,57*10^{- 6}*1,24}} = 8,97\]

\[k_{n1} = \left( 2n - 1 \right)\frac{\pi}{2} = \left( 2*1 - 1 \right)*\frac{3,14}{2} = 1,571\]

\[k_{n2} = 4,712\]

\[k_{n3} = 7,854\]

Данные для построения форм крутильных колебаний:

\begin{longtable}{@{}lllll@{}}
& x & \(\Phi_{1}(x)\) & \(\Phi_{2}(x)\) & \(\Phi_{3}(x)\)\\
0 & 0 & 0 & 0 & 0\\
0,1 & 0,115 & 1,612 & 4,627 & 7,046\\
0,2 & 0,23 & 3,17 & 7,927 & 8,722\\
0,3 & 0,345 & 4,627 & 8,956 & 3,755\\
0,4 & 0,46 & 5,932 & 7,419 & -4,072\\
0,5 & 0,575 & 7,046 & 3,755 & -8,797\\
0,6 & 0,69 & 7,927 & -0,984 & -6,821\\
0,7 & 0,805 & 8,552 & -5,442 & 0,352\\
0,8 & 0,92 & 8,899 & -8,34 & 7,257\\
0,9 & 1,035 & 8,956 & -8,849 & 8,634\\
1 & 1,15 & 8,722 & -6,821 & 3,433\\
\end{longtable}

\end{document}
